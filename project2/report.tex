\documentclass{article}
\usepackage[swedish]{babel}
\usepackage{booktabs, float, listings, mathtools, tabu}

\title{TANA21: Implementation sekantmetoden i Matlab}
\author{Emil Segerbäck \and Olav Övrebö}

\begin{document}
\maketitle
\newpage

\section{Inledning}
För att finna rötter till en funktion ($f(x) = 0$) finns det flera metoder man kan använda sig av. Sekantmetoden är en numerisk metod som söker rötter genom att approximera funktioner med en sekantlinje. Det skiljer den från Newtons metod som använder tangentlinjer. Fördelen med sekantmetoden är att den inte kräver att derivatan av $f$ är känd.

\section{Uppgift}
Sekantmetoden ska implementeras som en Matlab-funktion och analyseras för att avgöra dess konvergensordning.

\section{Teori}
beskrivning av numerisk metod och avbrottsvillkoret,

\section{Lösning}
(experimentbeskrivning),

\section{Resultat och svar}
För att testa den skrivna funktionen testades den för att hitta ett antal rötter till funktionerna $\sin(x)$ och $ (x+2) x (x-3)$. Resultaten redovisas i Tabell~\ref{tester}. Som tabellen visar gav alla de testade funktionerna ett resultat inom de efterfrågade feltoleranserna.
\begin{table}[H]
  \begin{tabu}{ l l l l X[1cm] X l X }
    Funktion & $x_0$ & $x_1$ & Tolerans & Förväntat resultat & Resultat $\approx$ & Erhållet fel $\leq$ \\
    \toprule
    $\sin(x)$ & 0.8 & 1 & 0.01 & 0 & $2.1544 \cdot 10^{-7}$ & $10^{-6}$ \\
    $\sin(x)$ & 4 & 3.8 & 0.01 & $\pi$ & 3.1416 & $10^{-9} $\\
    $ (x+2) x (x-3)$ & -1 & 1 & 0.1 & 0 & $2.8160 \cdot 10^{-4}$ & $10^{-3}$ \\
    $ (x+2) x (x-3)$ & 4 & 2 & 0.1 & 3 & 3 & $10^{-2}$ \\
    $ (x+2) x (x-3)$ & -10 & -5 & 0.1 & -2 & -2.0156 & $10^{-1}$ \\
  \end{tabu}
  \caption{ Resultat av testning av funktionen }\label{tester}
\end{table}

\section{Diskussion}

\end{document}
