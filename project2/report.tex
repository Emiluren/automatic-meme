\documentclass{report}
\usepackage{mathtools}

\title{TANA21: Implementation sekantmetoden i Matlab}
\author{Emil Segerbäck \and Olav Övrebö}

\begin{document}
\maketitle

\section{Inledning}
För att finna rötter till en funktion ($f(x) = 0$) finns det flera metoder man kan använda sig av. Sekantmetoden är en numerisk metod som söker rötter genom att approximera funktioner med en sekantlinje. Det skiljer den från Newtons metod som använder tangentlinjer. Fördelen med sekantmetoden är att den inte kräver att derivatan av $f$ är känd.

\section{Uppgift}
Sekantmetoden ska implementeras som en Matlab-funktion 

\section{Teori}
beskrivning av numerisk metod och avbrottsvillkoret,

\section{Lösning}
(experimentbeskrivning),

\section{Resultat och svar}

\section{Diskussion}

\end{document}
